
\begin{abstract}
Botnets are one of the primary threats in computer security today.  They are
used for launching denial of service attacks, sending spam and phishing emails,
and collecting private information.  However, every botnet requires
coordination.  In order to initiate an attack, a botmaster must communicate to
all of the bots in the network.  In this paper, we present a steganographic
system that demonstrates the feasibility of the social networking website Twitter
as a botnet command and control center that an attacker could use to reliably
communicate messages to a botnet with low latency and nearly perfect rate of
transmission.  Our system generates plausible
cover messages based on a required tweet length determined by an encoding map
that has been constructed based on the structure of the secret messages.  The
system considers both the input symbol frequencies (e.g.\ English letter
frequencies) as well as the tweet length posting frequencies for constructing
the encoding maps.  A technique for automatically generating Twitter account
names based on Markov chains is also presented so that the bots can connect to
new accounts if the existing botmaster account is unavailable.  All the experiments 
were performed using the $7.3M$ actual tweets from $3.7K$ verified accounts 
collected by the tweet parser developed by us. We have
evaluated the efficacy of the system using Emulab and usability of the system 
through Amazon's Mechanical Turk with promising results. An analysis of the steganographic security of the 
proposed system has also been provided. By demonstrating how a botmaster might 
perform such communication using online social networks, our work provides the basis
to detect and prevent emerging botnet activities.
\end{abstract}
