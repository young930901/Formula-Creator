\section{Conclusion}
\label{cha:conclusion}

In this paper, we have demonstrated a general-purpose stego system that allows secret communication through the
Twitter social network using only metadata for communication.  We have showed that this system can be used for
botnet command and control through the development of a Twitter stego system and a specialized botnet
C\&C language. 
We have discussed the performance evaluation of the proposed system using Emulab, 
and usability study using Amazon MTurk. We have also discussed how the vast 
userbase and large scale of Twitter facilitates ample steganograpic security.  
By demonstrating how a botmaster might perform
such communication using online social networks, our work provides the basis
to detect and prevent emerging botnet activities.

There are other possible steganographic techniques that can be applied using
Twitter.  For example, Twitter allows posting images along with
tweets\footnote{\url{https://support.twitter.com/articles/20156423-posting-photos-on-twitter}}
and there are many existing image steganography techniques \cite{watermarking}
that are often used for watermarking, but can also be used for communication.
The existing system can also be used on other social network websites such
as Facebook, but it may be necessary to collect data from these websites when
deciding on the message length distribution.  Unlike Twitter, these other
websites generally allow much longer posts, so the system could take advantage
of the increased variation.

It is possible to use this system for key exchange for other existing stego
systems.  It is necessary in steganographic key exchange to have a stego
system that can be used to transmit the key, otherwise an adversary can detect
the communication of the keys.  Because this system has a relatively low
information bandwidth, it may be well suited for key exchange that does not
require a significant amount of information.  This concept is applied in
cryptography where a public key algorithm such as RSA is used to send a
key for a symmetric algorithm such as AES because AES can achieve better
encryption and decryption performance than RSA, although it is technically
possible to send all of the communications using RSA.
