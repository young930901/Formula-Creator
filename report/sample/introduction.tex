\section{Introduction}
\label{cha:intro}

Computing and interconnectivity have spread through modern society as
electricity and plumbing have in the past, to become almost entirely ubiquitous.
Indeed, it is not uncommon for a single person to possess numerous computing
devices of varying power and portability, ranging from handheld smartphones and
tablets to notebooks and desktop computers.  Although these devices appear
different, they are all essentially the same.  They act as general purpose
computers that connect to the Internet to communicate with other devices across the
globe.

Cyber-criminals make use of this vast global Internet by installing or
convincing users to install malicious software, or \emph{malware}, on to their
devices that allow the criminals to control them remotely.  A collection of these
``zombie'' computers is called a \emph{botnet}.  Botnets are one of the
most prominent modern computer security threats \cite{botnet-taxonomy} and
are often used for various forms of cyber crime such as sending spam emails or
performing \emph{denial-of-service} (DoS) attacks against other computer
networks.  In fact, the botnet threat spreads beyond what we commonly refer to
as computing devices.  In the new \emph{internet-of-things} (IoT), many common household
appliances that contain embedded computers are being connected to the Internet.
A recent news story showed that these smart appliances, such as refrigerators,
were being used to distribute spam
email\footnote{\url{http://www.proofpoint.com/about-us/press-releases/01162014.php}}.

The design and communication patterns of these botnet can vary dramatically, as
they are created by cyber-criminals with the intent of hiding their presence.
Social networks have exploded in the past few years in the same way that the Internet
and the web before them.  Today, popular social networks like \emph{Twitter} and \emph{Facebook}
have hundreds of millions of users interacting and communicating in real time.
Even with the extremely large user bases, these services are rarely ever unavailable
and will transmit the communications at an incredible speed.  From this information,
a clever attacker will recognize that these networks are well suited for crafting
cyber attacks such as controlling an existing botnet.  They can take advantage of the
infrastructure, speed of transmission, and large userbase in which to hide to
control the bots.  This paper provides a proof of concept of this type of
botnet communication that is hidden within the social network \emph{Twitter}.

Understanding how attackers communicate with botnets is vital for botnet
defense.  If the attacker cannot coordinate the bots, they will be unable to
utilize the network.  As is common with computer security research, researching
both attack and defense can be useful.  Before a proper defense can be made,
new attacks must be understood.  Botnets are capable of attacking the
availability of a system using attacks such as DoS, where the
bot flood a network with requests to cause it to become
unresponsive to real traffic.  These attacks are especially dangerous because it
is difficult to distinguish between these fake requests and authentic
traffic trying to use the network.  In order to stop these attacks, it is best
to be able to cripple the botnet before it can begin.  Therefore, understanding
how an attacker might attempt to coordinate these bots is essential.

In this paper, our goal is to develop a method of coordinating bots in a
botnet that uses a \emph{stego-system} over a popular
social network, Twitter, using only meta data for communication.  This 
\emph{covert channel} can also be used
for arbitrary communication of relatively short messages outside of the realm of
botnet command and control.  By utilizing \emph{steganography}, we hide the existence of
the botnet control communication from the outside world while also utilizing
the power of the popular social networking website to ensure timely delivery of the
messages.

Towards that, we have first developed
a stego-system leveraging the Twitter social network
platform.  This system can be used for secret communication between various
parties for many domains.  Next, we have implemented a
botnet command and control (C\&C) communication system that utilizes the stego-system.
This C\&C system communicates entirely through the stego-system allowing
the botmaster to control each of the bots. We have also developed a technique 
for automatically generating Twitter account
names based on \emph{Markov chains} so that the bots can connect to
new accounts if the existing botmaster account is unavailable. Finally, we have evaluated
the efficacy and performance of both the stego-system and botnet C\&C.
All the experiments were performed using the $7.3M$ actual tweets from $3.7K$ 
verified accounts collected by the tweet parser developed by us. We have
evaluated the efficacy of the system using Emulab and usability of the system 
through Amazon's Mechanical
Turk with promising results. We have also provided a detailed steganographic 
security analysis of the proposed system. 

This paper is structured as follows: section \ref{cha:lit-review} contains
a broad literature review of the various techniques that are used in the
development of the research. Section
\ref{cha:methodology} describes the structure and implementation of the
{stego-system}, {botnet-cc}, and other components of the
system.  Section \ref{cha:evaluation} discusses the experiments
conducted and the evaluation of the results of the experiments including 
steganographic security analysis.  Section \ref{cha:conclusion} contains our concluding
remarks and future work related to the paper.
