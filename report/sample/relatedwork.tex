\section{Related Work}
\label{subsec:lit-review:botnets:related-work}

Stegobot \cite{stegobot} is a botnet designed to communicate using
social networks (specifically Facebook) and image stegangraphy.  The authors
design two separated types of message: bot commands and bot cargo.  The bot
commands are messages from the botmaster to the bots instructing them.  The bot
cargo are messages from the bots back to the botmaster containing stolen
information.

Stegobot \cite{stegobot} uses a distributed, peer to peer communication channel
and does not generate its own cover messages.  Instead, it uses the image files that the
victims are already uploading to the social network as the cover messages,
embedding the secret messages within.  The botnet software intercepts the images
being posted to embed the message before it is sent to Facebook.  Stegobot uses
the existing network of relationships for each victim as the communication
channel.  In their experiments, the authors used a set of 116 images.  One of
the difficulties of this technique is the automatic image manipulation performed
by Facebook as the images are uploaded.  This can tamper with the content of the
embedded message, requiring a highly robust stego system.

Natarajan et al. \cite{stegobot-detect} designed a detection scheme for Stegobot
that uses the information entropy of the image files that are acting as the
cover objects.  Their detection technique achieved average detection rates
exceeding 70\% in their experiments for several different image steganography
methods.

A similar work by Singh et al. \cite{twitter-botnet} uses Twitter for
botnet C\&C, but does not apply steganography to hide the communications.
Instead, the commands are posted directly to the Twitter account.  This allows
the botmaster to leverage the benefits of social networks for botnet C\&C, but
they used communication methods that will likely appear highly suspicious to any
viewers.

SocialClymene \cite{socialclymene} is a detection scheme designed specifically
for detecting stego-based botnet command and control methods using social networks,
however it is designed for image steganography such as stegobot.  CatchSync
\cite{catchsync} is another detection technique that looks at connectivity of
nodes in a directed graph to find suspicious nodes.  In the case of Twitter,
connectivity is determined by which accounts follow other accounts.

Sebastian et al. \cite{graybot} have created a similar mechanism for botnet command and control
using encrypted tweets.  This method mixes irrelevant sentences among tweets
that contain botnet commands.  Their command tweets follow the formula of
\ttf{\#keyword command}, where the value of \ttf{command} would be encrypted.
While this method can hide the commands being issued, it does not conceal the
existence of the commands.  Each command follows the same formula and can be
differentiated from other tweets posted on the botmaster's account.  Additionally,
no mechanism is described for recovering if the botmaster's account has been
closed due to detection of malicious activity. 
