%%This is a very basic article template.
%%There is just one section and two subsections.
\documentclass{article}

 \title{Senior Project Report}
 \author{Young Song}
 \date{\today}
 
 \begin{document}
 
 \maketitle

\section{Introduction}
This is an application that the user can manually input the numbers of variable
and operation. The created formula is usable for different values.
\section{Tokenize}
Each variable and operator is tokenized whenever the users choose and add to
their formulas. Each token has its type as final integer. Therefore, to check
its type, i just be able to call its name of type. The advantage of using token
class is that I do not need further tokenize for parsing. It saves much work.
\section{Parsing}

More plain text.
\section{Call Existing}
This function allows uses don't need to make same formula by adding varaible and
operator one by one to append or edit it. Users can just simply call the same
formula and append the formula to whatever they currently have. This can save
the time of users.
\section{Save and Retreive Data}
The first idea for save is to serialize the formul create class which is the
arralist of formula to the text or bin file. However, because, whenever
application restarts, it reseted the content of the text file, I rather used the
bundle to save the state of the application. In saved instance, I put the
arraylist of formula, and whenever the application restarts and the class is
created, if the saved instance is not empty, it calls the arraylist of formula
which is saved in saved instance.
\section{Challenge}
The most challenging part of this project was to implement parsing function. The
original idea of using parsing is to implement the CUP or ANTLR which create the
grammar rule for the parsing and automatically create the java file. However,
the tutorial file of CUP and ANTRL was not compiled for some reasons. So, I had
to manually create the parsing function. This was pretty challenging because I
had to convert the formula in post form and push it to the binary tree. Because
I had to do this from scratch, it took much time.
\section{Future Work}
\end{document}